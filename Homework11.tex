\documentclass[12pt]{article}
 \usepackage[margin=1in]{geometry} 
\usepackage{amsmath,amsthm,amssymb,amsfonts}
\usepackage{stackengine}
 
\newcommand{\longdiv}[2]{$#1\overline{\smash{)}#2}$}
\newcommand{\N}{\mathbb{N}}
\newcommand{\Z}{\mathbb{Z}}
 
\newenvironment{problem}[2][Problem]{\begin{trivlist}
\item[\hskip \labelsep {\bfseries #1}\hskip \labelsep {\bfseries #2.}]}{\end{trivlist}}

\begin{document}

\title{Pg. 201-202, #1, 2-28 even}
\date{November 13, 2016}
\author{Mia Jones}

\maketitle

\begin{problem}{1}
If the results of long division repeat without end, the quotient is a \boxed{\text{repeating decimal.}}
\end{problem}

\begin{problem}{2}
If a long division problem gives a remainder of 0, the quotient is a \boxed{\text{finite decimal.}}
\end{problem}

\begin{problem}{4}
1 $\frac{1}{4}$ = \boxed{1.25}
\end{problem}

\begin{problem}{6}
$\frac{5}{6}$ = \boxed{0.8\overline{3}} 
\\
\def\stackalignment{r}

\savestack{\workdetails}{\Longunderstack{\longdiv{6}{5.00}
  \underline{4.8\phantom{0}} 
  \phantom{0}20
  \underline{\phantom{0}18}
  \phantom{00}2
}}

\Shortstack{\boxed{0.8\overline{3}} {\workdetails}}
\end{problem}

\begin{problem}{8}
2 $\frac{2}{3}$ = \boxed{2.\overline{6}}
\end{problem}

\begin{problem}{10}
5 $\frac{4}{25}$ = \boxed{5.16}
\end{problem}

\begin{problem}{12}
9 $\frac{7}{111}$ = \boxed{9.0\overline{360}}
\\
\def\stackalignment{r}

\savestack{\workdetails}{\Longunderstack{\longdiv{111}{7.00000}
  \underline{666\phantom{000}} 
  \phantom{70}340\phantom{00}
  \underline{333\phantom{00}}
  \phantom{00}700
}}

\Shortstack{\boxed{0.0\overline{630}} {\workdetails}}

\end{problem}

\begin{problem}{14}
\boxed{0.\overline{7}}
\end{problem}

\begin{problem}{16}
\boxed{3.5\overline{8}}
\end{problem}

\begin{problem}{18}
\def\stackalignment{r}

\savestack{\workdetails}{\Longunderstack{\longdiv{11}{5.00}
  \underline{44\phantom{0}} 
  \phantom{5}60
  \underline{\phantom{0}55}
  \phantom{00}5
}}

\Shortstack{\boxed{0.\overline{45}} {\workdetails}}

Both decimal digits shown will repeat, so the bar should go over the 4 and the 5: \boxed{0.\overline{45.}} 
\end{problem}

\begin{problem}{20}
0.8 = $\boxed{\frac{4}{5}}$
\end{problem}

\begin{problem}{22}
0.125 = $\boxed{\frac{1}{8}}$
\end{problem}

\begin{problem}{24}
4.175 = $\boxed{4 \frac{7}{40}}$
\end{problem}

\begin{problem}{26}

\end{problem}
\end{document}