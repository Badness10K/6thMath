\documentclass[12pt]{article}
 \usepackage[margin=1in]{geometry} 
\usepackage{amsmath,amsthm,amssymb,amsfonts}
\usepackage{stackengine}
 
\newcommand{\longdiv}[2]{$#1\overline{\smash{)}#2}$}
\newcommand{\N}{\mathbb{N}}
\newcommand{\Z}{\mathbb{Z}}
 
\newenvironment{problem}[2][Problem]{\begin{trivlist}
\item[\hskip \labelsep {\bfseries #1}\hskip \labelsep {\bfseries #2.}]}{\end{trivlist}}

\title{Pg. 228-229, #1-15}
\author{Mia Jones}
\date{November 27 2018}

\begin{document}


\maketitle

\begin{problem}{1}
A proper fraction has a numerator less than its denominator, whereas an improper fraction has a numerator greater than its denominator.
\end{problem}

\begin{problem}{2}
If the fractional part of the second mixed number is greater, rename the fraction to subtract.
\end{problem}

\begin{problem}{3}
$12 \frac{3}{5} + 5 \frac{1}{5} = \boxed{17 \frac{4}{5}.}$ \\ 
$\approx 13 + 5 = \underline{18.}$
\end{problem}

\begin{problem}{4}
$22 \frac{2}{7} + 14 \frac{4}{7} = \boxed{39 \frac{6}{7}.}$ \\
$\approx 22 + 18 = \underline{40.}$
\end{problem}

\begin{problem}{5}
$8 \frac{7}{12} + 4 \frac{5}{12} = 12 \frac{12}{12} = \boxed{13.}$ \\
$\approx 9 + 4 = \underline{13.}$
\end{problem}

\begin{problem}{6}
$8 \frac{3}{4} + 2 \frac{3}{4} = 10 \frac{6}{4} = \boxed{11 \frac{1}{2}.}$ \\
$\approx 9 + 3 = \underline{12.}$
\end{problem}

\begin{problem}{7}
$3 \frac{2}{3} - 2 \frac{1}{3} = \boxed{1 \frac{1}{3}.}$ \\
$\approx 4 - 2 = \underline{2.}$
\end{problem}

\begin{problem}{8}
$7 \frac{3}{5} - 3 \frac{1}{5} = \boxed{4 \frac{2}{5}.}$ \\
$\approx 8 - 3 = \underline{5.}$
\end{problem}

\begin{problem}{9}
$8 \frac{4}{9} - 5 \frac{2}{9} = \boxed{3 \frac{2}{9}.}$ \\
$\approx 8 - 5 = \underline{3.}$
\end{problem}

\begin{problem}{10}
$13 \frac{5}{6} - 9 \frac{1}{6} = \boxed{4 \frac{2}{3}.}$ \\
$\approx 14 - 9 = \underline{5.}$
\end{problem}

\begin{problem}{11}
$4 \frac{1}{4} + 3 \frac{3}{8} = 4 \frac{2}{8} + 3 \frac{3}{8} = \boxed{7 \frac{5}{8}.}$ \\
$\approx 4 + 3 = \underline{7.}$
\end{problem}

\begin{problem}{12}
$3 \frac{2}{3} + 8 \frac{1}{6} = 3 \frac{4}{6} + 8 \frac{1}{6} = \boxed{11 \frac{5}{6}.}$ \\
$\approx 4 + 8 = \underline{12.}$
\end{problem}

\begin{problem}{13}
$4 \frac{3}{4} + 6 \frac{2}{3} = 4 \frac{9}{12} + 6 \frac{8}{12} = 10 \frac{17}{12} = \boxed{11 \frac{5}{12}.}$ \\
$\approx 5 + 6 = \underline{11.}$
\end{problem}

\begin{problem}{14}
$5 \frac{1}{4} + 2 \frac{5}{6} = 5 \frac{3}{12} + 2 \frac{10}{12} = 7 \frac{13}{12} = \boxed{8 \frac{1}{12}.}$ \\
$\approx 5 + 3 = \underline{8.}$
\end{problem}

\begin{problem}{15}
$6 \frac{2}{5} + 11 \frac{1}{6} = 6 \frac{12}{30} + 11 \frac{5}{30} = \boxed{17 \frac{17}{30}.}$ \\
\approx 6 + 11 = \underline{17.}
\end{problem}

\end{document}