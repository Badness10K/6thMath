\documentclass[12pt]{article}
 \usepackage[margin=1in]{geometry} 
\usepackage{amsmath,amsthm,amssymb,amsfonts}
\usepackage{stackengine}
 \usepackage[makeroom]{cancel}
\newcommand{\longdiv}[2]{$#1\overline{\smash{)}#2}$}
\newcommand{\N}{\mathbb{N}}
\newcommand{\Z}{\mathbb{Z}}
 
\newenvironment{problem}[2][Problem]{\begin{trivlist}
\item[\hskip \labelsep {\bfseries #1}\hskip \labelsep {\bfseries #2.}]}{\end{trivlist}}

\title{Pg. 248-249, #1, 2, 7-15, 26-30}
\author{Mia Jones}
\date{December 5 2018}

\begin{document}


\maketitle


\begin{problem}{1}
\boxed{\text{Cup}}
\end{problem}

\begin{problem}{2}
\boxed{\text{Ton}}
\end{problem}

\begin{problem}{7}
$\boxed{4 \frac{1}{2} \text{lb}}$
\end{problem}

\begin{problem}{8}
$\boxed{\frac{3}{8} \text{lb}}$
\end{problem}

\begin{problem}{9}
A newborn baby weighs $\boxed{7 \frac{1}{2} \text{ounces.}}$
\end{problem}

\begin{problem}{10}
A tennis racket weighs $\boxed{9\frac{1}{2} \text{ounces.}}$
\end{problem}

\begin{problem}{11}
A shampoo bottle holds \boxed{12 \text{ ounces.}}
\end{problem}

\begin{problem}{12}
 A tea kettle holds about \boxed{12 \text{ ounces.}}
\end{problem}

\begin{problem}{13}
A mug’s $\mathbf{capacity}$ is 16 fluid ounces. The weight of an object is how heavy it is, and the capacity of the object is a measurement of how much it can carry.
\end{problem}

\begin{problem}{14}
$\boxed{2 \frac{1}{4} \text{cups.}}$
\end{problem}

\begin{problem}{15}
$\boxed{1 \frac{1}{2} \text{cups.}}$
\end{problem}

\begin{problem}{26}
We would want to use a tape measure, because it can stretch along the length of the wall appropriately. A yardstick or a ruler would certainly be too small to measure the wall.
\end{problem}

\begin{problem}{27}
D. 800 mi.
\end{problem}

\begin{problem}{28}
A $3 \frac{1}{2} \text{tons.}$
\end{problem}

\begin{problem}{29}
Dave is right. A pair of two pound sunglasses would be much too heavy to carry on someone's face! Sunglasses are light, so 2 oz. makes sense.
\end{problem}

\begin{problem}{30}
The scale with the tenths increments will be more precise, as the increments are smaller and therefore the hand will point to an estimate closer to the real weight. 
\end{problem}

\end{document}