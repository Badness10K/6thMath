\documentclass[12pt]{article}
 \usepackage[margin=1in]{geometry} 
\usepackage{amsmath,amsthm,amssymb,amsfonts, gensymb}
\usepackage{stackengine}
 \usepackage[makeroom]{cancel}
\newcommand{\longdiv}[2]{$#1\overline{\smash{)}#2}$}
\newcommand{\N}{\mathbb{N}}
\newcommand{\Z}{\mathbb{Z}}
 
\newenvironment{problem}[2][Problem]{\begin{trivlist}
\item[\hskip \labelsep {\bfseries #1}\hskip \labelsep {\bfseries #2.}]}{\end{trivlist}}

\title{p. 309 #1,2-8 even, 11-21 odd, 26-34 even}
\date{January 13 2019}
\author{Mia Jones}

\begin{document}

\maketitle

\begin{problem}{1}
You can use the distributive property to write \underline{equivalent} expressions.
\end{problem}

\begin{problem}{2}
\begin{align*}
    &\text{Distributive Property}       &\text{Check} \\
    &5(3 + 7)        &5(3 + 7) \\
    = &5(3) + 5(7)   &=5(10) \\
    = &15 + 35       &=\underline{50.} \\
    = &\boxed{50.}   &
\end{align*}
\end{problem}

\begin{problem}{4}
\begin{align*}
    &\text{Distributive Property}  &\text{Check} \\
    &7(3) + 7(4)    &7(3) + 7(4) \\
    = &7(3 + 4)    &= 21 + 28 \\
    = &7(7)        &= \underline{49.} \\
    = &\boxed{49.} 
\end{align*}
\end{problem}

\begin{problem}{6}
\begin{align*}
    &\text{Distributive Property}  &\text{Check} \\
    &6(\frac{5}{12}) - 6(\frac{1}{12}) &6(\frac{5}{12})-6(\frac{1}{12})\\ = &6(\frac{5}{12} - \frac{1}{12})    &=\frac{30}{12}-\frac{6}{12}\\
    = &6(\frac{1}{3})       &=\frac{24}{12} \\
    = &\boxed{2.}       &\underline{2.}
\end{align*}
\end{problem}

\begin{problem}{8}
B.
\end{problem}

\begin{problem}{11}
6(16.85) \\
= 6(16 + 0.85) \\
= 6(16) + 6(0.85) \\
= 96 + 5.1 \\
= \boxed{101.1.}
\end{problem}

\begin{problem}{13}
3(7.3) + 3(2.7) \\
= 3(7.3 + 2.7) \\
= 3(10) \\
= \boxed{30.}
\end{problem}

\begin{problem}{15}
9(13.2) + 9(6.8) \\
= 9(13.2 + 6.8) \\
= 9(20) \\
= \boxed{180.}
\end{problem}

\begin{problem}{17}
$13(\frac{3}{7}) - 13(-\frac{4}{7}) \\
=13(\frac{3}{7} - -\frac{4}{7}) \\
=13(\frac{3}{7} + \frac{4}{7}) \\
=13(1) \\
=\boxed{13.}$
\end{problem}

\begin{problem}{19}
We should have applied the negative sign to 4, making the sum \boxed{-72.}
\end{problem}

\begin{problem}{21}
\begin{align*}
    &\text{Equation 1} &\text{Equation 2} \\
    &8(3) + 14 + 8(-4)  &8(3) + 14 + 8(-4) \\
    = &8(3 + -4) + 14 &= 24 + 14 + -32 \\
    = &8(-1) + 14 &= 38 + -32 \\
    = -8 + 14 &= 8 \\
    = 
\end{align*}
\end{problem}


\end{document}