\documentclass[12pt]{article}
 \usepackage[margin=1in]{geometry} 
\usepackage{amsmath,amsthm,amssymb,amsfonts}
 
\newcommand{\N}{\mathbb{N}}
\newcommand{\Z}{\mathbb{Z}}
 
\newenvironment{problem}[2][Problem]{\begin{trivlist}
\item[\hskip \labelsep {\bfseries #1}\hskip \labelsep {\bfseries #2.}]}{\end{trivlist}}


 \begin{document}

\title{Pg. 167-168, #1, 2-20 even}
\author{Mia Jones}
\date{October 31 2018}

\maketitle


 \begin{problem}{1}
 To factor a whole number as a product of prime factors is called \boxed{\text{prime factorization.}}
 \end{problem}
 
\begin{problem}{2}
A whole number greater than 1 that has whole number factors other than 1 and itself, such as 22, is called a \boxed{\text{composite}} number.
\end{problem} 

 \begin{problem}{4}
25: \boxed{1, 5, 25} 
 \end{problem}
 
 \begin{problem}{6}
 84: \boxed{1, 2, 3, 4, 6, 7, 12, 18, 21, 28, 42, 84}
 \end{problem}
 
 \begin{problem}{8}
88 is composite, because it is divisible by many other factors(Ex: 2), in addition to 1 and itself.
 \end{problem}
 
 \begin{problem}{10}
 39 is composite, because it is divisible by other factors(Ex. 3) as well as 1 and itself.
 \end{problem}
 
 \begin{problem}{12}
 61 is prime, because it is only divisible by 1 and itself.
 \end{problem}
 
 \begin{problem}{14}
41 is prime, because it cannot be divided by any integers, excepting 1 and itself. 
 \end{problem}

\begin{problem}{16}
201 is composite, because it is divisible by 3 in addition to being divisible by 1 and 201.
\end{problem}

\begin{problem}{18}

\end{problem}

 \end{document}