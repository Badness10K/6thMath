\documentclass[12pt]{article}
 \usepackage[margin=1in]{geometry} 
\usepackage{amsmath,amsthm,amssymb,amsfonts}
 
\newcommand{\N}{\mathbb{N}}
\newcommand{\Z}{\mathbb{Z}}
 
\newenvironment{problem}[2][Problem]{\begin{trivlist}
\item[\hskip \labelsep {\bfseries #1}\hskip \labelsep {\bfseries #2.}]}{\end{trivlist}}

\begin{document}

\title{Pg. 178-179, #1, 2, 11-21 odd, 22, 27-30, 40, 41}
\date{November 5 2018}
\author{Mia Jones}

\maketitle


\begin{problem}{1}
If a fraction's GCF is 1, the fraction is in simplest form.
\end{problem}

\begin{problem}{2}
Fractions that represent the same part-to-whole relationship are called \\ \boxed{\text{equivalent fractions.}}
\end{problem}

\begin{problem}{11}
\[\frac{32}{72} = \boxed{\frac{4}{9}.}\]
\end{problem}\

\begin{problem}{13}
\[\frac{15}{21} = \boxed{\frac{5}{7}.}\]
\end{problem}

\begin{problem}{15}
\[\frac{28}{48} = \boxed{\frac{7}{24}.}\]
\end{problem}

\begin{problem}{17}
\[\frac{24}{32} = \boxed{\frac{2}{3}.}\]
\end{problem}

\begin{problem}{19}
\[\frac{49}{105} = \boxed{\frac{7}{37}.}\]
\end{problem}

\begin{problem}{21}
This fraction has a GCF of 2. We can divide by 2 to put this fraction in simplest form: \\ \[\frac{8}{14} = \boxed{\frac{4}{7}.}\]
\end{problem}

\begin{problem}{22}
\[\frac{14}{21} \boxed{=} \frac{24}{36}.\]
\end{problem}

\begin{problem}{27}
\[\frac{18}{24} = \frac{\boxed{9}}{12}.\]
\end{problem}

\begin{problem}{28}
\[\frac{12}{21} = \frac{\boxed{4}}{7}.\]
\end{problem}

\begin{problem}{29}
\[\frac{3}{7} = \frac{18}{\boxed{42}}.\]
\end{problem}

\begin{problem}{30}
\[\frac{12}{16} = \frac{6}{\boxed{8}}.\]
\end{problem}

\begin{problem}{40}
\[\frac{6}{22} = \boxed{\frac{3}{11}}.\]
\end{problem}

\begin{problem}{41}
\[\frac{20}{50} = \frac{2}{5}\]  \boxed{B}
\end{problem}
\end{document}